\documentclass[UTF8]{article}
\usepackage{ctex}
\usepackage{amsmath}

% 2 Chapter Text Layout   

\author{I Am Your Father}
\title{Test Article}


\begin{document}
	
	\section{Test}
	\ldots when Einstein introduced his formula
	\begin{equation}
		e = m \cdot c^2 \; , 
	\end{equation}

	which is at the same time the most widely known and the least well understand physical formula. 
	
	\ldots from which follows Kirchhoff's current law:
	\begin{equation}
		\sum_{k=1}^{n} I_k =0 \; .
	\end{equation}

	Kirchhoff's voltage law can be derived \ldots 
	
	
	\ldots which has several advantage. 
	
	\begin{equation}
		I_D = I_F - I_R 
	\end{equation}
	is the core of a very different transistor model. \ldots 
	
	``Please press the 'x' key.''
	
	daughter-in-law, X-rated\\
	pages 13--67\\
	yes--or no? \\
	$0$, $1$ and $-1$
	
	It's $-30\,^{\circ} \mathrm{C}$.
	I will soon start to super-conduct. 
	
	Not like this ... but like this:\\ New York, Tokyo, 
	Budapest, \ldots
	
	Mr. Smith was happy to see her\\
	cf. Fig.5 \\
	I like BASIC\@. What about you?\par 
	\par
	
	A reference to this subsection \label{sec:this} looks like:
	``see section \ref{sec:this} on page \pageref{sec:this}	.''
	
	\underline{text}
	\emph{text}
	
	\emph{If you use emphasizing inside a piece of emphasized text, then \LaTeX{} uses the \emph{normal} font for emphasizing. }
	
	\textit{You can also \emph{emphasize} text if it is set in italics,}
	

	\textsf{ in a \emph{ sans-serif} font,}
	
	\texttt{or in \emph{typewriter} style.}
	
	\flushleft
	\begin{enumerate}
		\item You can mix the list environments to your taste:
		\begin{itemize}
			\item But it might start to look silly. 
			\item[-] With a dash.
		\end{itemize}
	
		\item Therefore remember:
			\begin{description}
				\item[Stupid] things will not become smart because they are in a list.
				\item[Smart] things, though, can be presented beautifully in a list.
			\end{description}
	\end{enumerate}
	
	
	\begin{flushleft}
		This text is \\ left-aligned.
		\LaTeX{} is not trying to \\make each line the same length.
	\end{flushleft}
	
		
	\begin{flushleft}
		This text is \\ left-aligned.
		\LaTeX{} is not trying to \par make each line the same length.
	\end{flushleft}
	
	
	This text is \\ left-aligned.\par
	\indent \\LaTeX{} is not trying to make each line the same length.
	
	\begin{flushright}
		This text is right-\\aligned.
		\LaTeX{} is not trying to make each line the same length.
	\end{flushright}
	
	\begin{center}
		At the centre\\ of the earth
	\end{center}

	
	
\end{document}